\message{: version 27feb04}%27mar03%29aug00%02mar00%28sep99%08jun97%15apr97%20mar97%17feb97%18dec96
%%%%%%%%%%%%%%%%%%%%%%%%%%%%%%%%%%%  refmac.tex
% copyright 1995 Carl de Boor

%  \input this file AFTER the body of your paper and PRIOR to the references.
%
%  This file, for use with the splinebib references, describes those macros 
% one is likely to customize, for local use. Many examples are given, with 
% the active default the so-called harvard style. Other styles, such as
%    bibstyle, SIAMstyle, CAstyle, NMstyle, EJAstyle, JATstyle, AMSstyle,
%    Baltzerstyle, Chamonixstyle, plainstyle, MMMASstyle, BITstyle
%    BibTeX, Kluwerstyle 
% can be activated by saying, e.g.,
%    \def\SIAMstyle{}
%    \input refmac
% i.e., by defining the appropriate macro BEFORE \input'ing the present file.
%
% If you are interested in having a stand-alone TeX-file containing the
% references in the chosen style, preface the \input refmac statement by
%    \def\echobib{}
% This will cause the file <jobname>.bib to be written with all the
% references in plain TeX, each starting off with \nextref{...}, with  ...
% empty unless \rl is defined.  This makes it possible, in a TeX file to be 
% submitted to a journal, to replace the bibliographic material using the 
% splinebib stuff by an 
%    \input <jobname>.bib 
% prefaced by an appropriate definition of the macro \nextref.

% Be sure to set  \locbib  below to point to the directory that contains
% the needed files  journal.tex  and  proceed.tex  that contain the
% the splinebib macros the user is not likely to change. 

%%%%%%%%%%%%%%%%%%%  where are the files ????   %%%%%%%%%%%%%%%%%%%%%%%%%%%%
%
%  Specify the directory which contains the standard files  journal.tex  and
%  proceed.tex  and, perhaps, files of customized definitions:

\def\locbib{}       %         <<<<<<<   PLEASE SET THIS

\input \locbib journal

%%%%%%%%%%%%%%%%%%%%  is  proceed.tex  not going to be used ???? %%%%%%%%%%%%
%
%  The macro  \lookupp  is used in the processing of any reference involving a
% standard proceedings (i.e., any \refP), with  \lookupp  ferreting out the 
% needed information from the file  proceed.tex . If the needed proceedings
% information is defined directly prior to the references needing it, then 
% \lookupp  needs to be disabled, by defining it here as
%    \def\lookupp#1{}

%%%%%%%%%%% what should happen at the beginning of every reference ???  %%%%
%
%  The macro  \nextref  should be customized; it can be used to determine the 
% placement of the individual references, including spacing between references, 
% and the style in which the reference label is displayed. For this, it relies 
% on one or both of the macros  \rh  and \rl , with \rh  the reference *HANDLE* 
% and \rl the reference *LABEL* for the reference, meaning that the former is a 
% convenient name for the reference while the latter is what is printed when 
% the handle or name is used. The examples below rely on the following macro
%  \rhl  from de Boor's /p/ftp/Approx/carlformat.tex file:
%
\ifx\rhl\undefined
\def\rhl#1{\rahel#1::/}
\def\rahel#1:#2:#3/{\xdef\rh{#1}\def\next{#2}\ifx\next\empty\xdef\rl{#1}\else\xdef\rl{#2}\fi}
\fi
%
% For example, the choice
%    \def\nextref{\item{[\rl]}}
% expects the reference LABEL \rl  for each reference to be defined prior to 
% execution of the specific \ref[BDJPQR]  macro,  e.g., as in
%    \rhl{L66}
%    \refB Lorentz, G.G.;
%    Approximation of Functions;
%    Holt, Rinehart and Winston, New York; 1966;
% and places that label, enclosed in brackets, into the left margin.
%
%  If the  \cite  mechanism from de Boor's carlformat.tex file is to be used,
% \nextref  could have the following form
%    \def\nextref{\item{\defcite{\rh}}}
% which, e.g.,  permits reference to the above item by saying  \cite{L66} .
% If it is desirable to have this book referred to by saying
%  \cite{Lorentzbook2} , yet have the printed label be  [L66] , use instead
%     \rhl{Lorentzbook2:L66}
%    \refB Lorentz, G.G.;
%     ...

% Specification of the reference HANDLE  \rh  is also needed if the 
% references are to be serialized, in the unfortunate standard form. In this 
% case, any explicit definition of the reference LABEL  \rl  will be ignored
% since its value is automatically supplied as the next number, in the 
% following version of  \nextref : 
% 
\newcount\refnum\refnum0
% \def\nextref{\global\advance\refnum by 1 \def\rl{\the\refnum}%
% \item{\defcite{\rl}}}

%%%%%%%%%%%%%%  what should be the exact form of the references ????  %%%%%%
%
% The macros \nextref, \formfirstauthor, ..., \form[BDJPR]  should be customized. 
% 
% Some styles make use of the small caps font:
\font\cmsc=cmcsc10 %%%% small caps font

%%%%%%%%%%%%%%%%%    the present DEFAULT style is the harvard style:

%% %%%%%%%%%%%%%%%%%%%%%  harvardstyle (except for formfirstauthor)
%
% The \nextref  here does nothing more than start a new item, since, in the
% Harvard style, the author(s)' name, followed by year of publication, serves
% as complete identifier of a reference. 
% This is overridden by any prior definition of  \nextref , unless there was
% an explicit call for the harvard style (by saying \def\harvardstyle{} 
% before input'ing  refmac.tex ).

\ifx\nextref\undefined\gdef\nextref{\item{}}\else%
\ifx\harvardstyle\undefined\else\gdef\nextref{\item{}}\fi\fi

% Here is the specification of the four possible author-fields:

% \def\formfirstauthor{\the\lastname, \the\firstname} %<< proper harvard style
\def\formfirstauthor{\the\firstname\the\lastname}
\def\formnextauthor{, \the\firstname\the\lastname}
\def\formotherauthor{ and \the\firstname\the\lastname}
\def\formlastauthor{,\formotherauthor}

% Here are the five basic reference types:

\def\formB{\nextref\the\au\ (\yr), {\it \the\ti\/}, \the\pb\ (\the\pl).
\ifechobib{\echostart
\ww{\the\au\ (\yr),}%
\ww{{\noexpand\it \the\ti\noexpand\/},}%
\ww{\the\pb\ (\the\pl).}}\fi}

\def\formD{\nextref\the\au\ (\yr), ``\the\ti'', dissertation, \the\pl.
\ifechobib{\echostart%
\ww{\the\au\ (\yr),}%
\ww{``\the\ti'',}%
\ww{dissertation, \the\pl.}}\fi}

\def\formJ{\nextref\the\au\ (\yr), ``\the\ti'', {\it\the\jr\/} {\bf\vl}, \pp.
\ifechobib{\echostart%
\ww{\the\au\ (\yr),}%
\ww{``\the\ti'',}%
\ww{{\noexpand\it\the\jr\noexpand\/} {\noexpand\bf\vl}, \pp.}}\fi}

\def\formP{\nextref\the\au\ (\yr), ``\the\ti'', in {\it\the\tit\/}
(\the\aut, ed\edsop), \the\pub\ (\the\pl), \pp.
\ifechobib{\echostart%
\ww{\the\au\ (\yr),}%
\ww{``\the\ti'',}%
\ww{in {\noexpand\it\the\tit\noexpand\/}}%
\ww{(\the\aut, ed\edsop),}%
\ww{\the\pub\ (\the\pl), \pp.}}\fi}

\def\formR{\nextref\the\au\ (\yr), ``\the\ti''\ifx\is\empty\else, \fi\is.
\ifechobib{\echostart%
\ww{\the\au\ (\yr),}%
\ifx\is\empty\ww{``\the\ti''.}\else
\ww{``\the\ti'',}\ww{\is.}\fi}\fi}
%%%%%%%%%%%%%%%%%%%%%%%%%%%%%%%%%%%%%%%%%%%%%%%   end of harvard style

\ifx\bibstyle\undefined\else{ %%%%%%%%%%%%%%%%%%%%%%%%%%%   bibstyle
% This is more or less the style in which the 
% references themselves are given; see the
% definitions of  \ref[A-Z] 
% given in  journal.tex . For this reason, \jobname.bib writing is not enabled.

\gdef\formfirstauthor{\the\lastname, \the\firstname}
\gdef\formnextauthor{,\the\lastname, \the\firstname}
\gdef\formotherauthor{ and \the\firstname\the\lastname}
\gdef\formlastauthor{,\formotherauthor}

% Here are the five basic reference types:

\gdef\formB{\nextref\the\au, {\sl \the\ti\/}, \the\pb\ (\the\pl), \yr.}
\gdef\formD{\nextref\the\au, \the\ti, dissertation, \the\pl, \yr.} 
\gdef\formJ{\nextref\the\au, \the\ti, \the\jr\ {\bf\vl} (\yr), \pp.}
\gdef\formP{\nextref\the\au, \the\ti, in {\sl\the\tit\/}, \the\aut{}%
(ed\edsop.), \the\pub\ (\the\pl), (\yr), \pp.} 
\gdef\formR{\nextref\the\au, \the\ti, \ifx\is\empty\else\is,\fi \yr.}
}\fi %%%%%%%%%%%%%%%%%%%%%%%%%%%%%%%%%%%%%%%%%   end of bibstyle

% Some of the next styles may need to know whether \jobname.bib is to be
% written, so, here is all the material connected with that:

%%%%%%%%%%%%%%%%%%%%%%%%%%%%%%%%%%%%%%%%%%%%%%
%
%   initiating the STAND-ALONE file  <jobname>.bib
%

\newif\ifechobib\ifx\echobib\undefined\else\echobibtrue\fi

\newwrite\w
\def\ww#1{\immediate\write\w{#1}}
% Material other than the references can be passed on as well with the aid of
\newtoks\percent{\catcode`\%=12\relax\global\percent={%}}
\newtoks\sharp{\catcode`\#=12\relax\global\sharp={#}}
\newtoks\oneline
\def\asis#1{{\oneline={#1}\ww{\the\oneline}}}
% for passing lines (encased in \asis{...}) verbatim into the file
% having backslashed any % signs in the process, hence having to edit
% the resulting file to remove those backslashes. 
% This editing can be avoided by replacing any %<material> by \%{<material>}
% (which causes each such %<material> to appear on a separate line in .bib).

\ifx\BibTeX\undefined\else\echobibtrue\fi % make sure that BibTeX works

\ifechobib{
\immediate\openout\w=\jobname.bib
\gdef\echostart{\ww{ }%
\ifx\rl\undefined\ww{\noexpand\nextref{}}\else\ww{\noexpand\nextref{\rl}}\fi}
\gdef\%#1{\oneline={#1}\ww{\the\percent\the\oneline}}
\ww{\noexpand\def\noexpand\nextref\the\sharp1{\noexpand\medskip\noexpand\item{\noexpand\bf[\the\sharp1]}\noexpand\quad}}
\ww{\the\percent standard macros} 
\asis{\def\RR{\mathop{{\rm I}\kern-.16em{\rm R}}\nolimits}}
\asis{\def\CC{\hbox{\rm C\kern -.58em {\raise .54ex \hbox{$\scriptscriptstyle |$}}}
 \kern-.55em {\raise .53ex \hbox{$\scriptscriptstyle |$}} }}
\asis{\def\ZZ{\mathop{{\rm Z}\kern-.28em{\rm Z}}\nolimits}}
\asis{\def\semicolon{;}
}}\fi
% The idea is to generate, in \jobname.bib , the explicit references, in a
% particular style, but usable in a TeX file  without any further reference 
% to journal.tex, proceed.tex, or refmac.tex .
%%%%%%%%%%%%%%%%%%%%%%%%%%%%%%%%%%%% end of STAND-ALONE \jobname.bib material

\newtoks\bsbs\bsbs={\\}
\ifx\BITstyle\undefined\else{%%%%%%%%%%%%%%% This defines the BIT style
%%    aug00, by inspection of their sample2e.tex file

\gdef\formfirstauthor{\the\firstname\the\lastname}
\gdef\formnextauthor{, \the\firstname\the\lastname}
\gdef\formotherauthor{ and \the\firstname\the\lastname}
\gdef\formlastauthor{,\formotherauthor}

\gdef\formB{\nextref{\the\au}, {\it\the\ti\/}, \the\pb, \the\pl, \yr.
\ifechobib{\echostart%
\ww{{\the\au},}%
\ww{{\noexpand\it\the\ti\noexpand\/},}%
\ww{\the\pb,}%
\ww{\the\pl,}%
\ww{\yr.}%
\ww{\noexpand\endref}}\fi}

\gdef\formD{\nextref{\the\au}, {\it\the\ti\/}, Ph.D.~thesis, \the\pl, \yr.
\ifechobib{\echostart%
\ww{{\the\au},}%
\ww{{\noexpand\it\the\ti\noexpand\/},}%
\ww{Ph.D.~thesis,}%
\ww{\the\pl,}%
\ww{\yr.}
\ww{\noexpand\endref}}\fi}

\gdef\formJ{\nextref{\the\au}, {\it\the\ti\/}, \the\jr, \vl\ (\yr), pp.~\pp.
\ifechobib{\echostart%
\ww{{\the\au},}%
\ww{{\noexpand\it\the\ti\noexpand\/},}%
\ww{\the\jr, \vl\ (\yr), pp.\noexpand~\pp.}
\ww{\noexpand\endref}}\fi}

\gdef\formP{\nextref{\the\au}, {\it\the\ti\/}, in \the\tit\ 
\the\aut, ed\edsop, \the\pub, \the\pl, \yr, pp.~\pp. 
\ifechobib{\echostart%
\ww{{\the\au},}%
\ww{{\noexpand\it\the\ti\noexpand\/},}%
\ww{in \the\tit,}%
\ww{\the\aut, ed\edsop,}%
\ww{\the\pub,}%
\ww{\the\pl,}%
\ww{\yr, pp.\noexpand~\pp.}
\ww{\noexpand\endref}}\fi}

\gdef\formR{\nextref{\the\au}, {\it\the\ti\/}, \ifx\is\empty\else\is, \fi \yr.
\ifechobib{\echostart%
\ww{{\the\au},}%
\ww{{\noexpand\it\the\ti\noexpand\/},}%
\ifx\is\empty\ww{\yr.}\else\ww{\is, \yr.}\fi%
\ww{\noexpand\endref}}\fi}
}\fi %%%%%%%%%%%%%%%%%%%%%%%%%%%%%%%%%%% end of definition of the BIT style

\newtoks\bsbs\bsbs={\\}
\ifx\SIAMstyle\undefined\else{%%%%%%%%%%%%%%% This defines the SIAM style
%%    jun 95/adapted from shayne waldron's missive/cb

\gdef\formfirstauthor{\the\firstname\the\lastname}
\gdef\formnextauthor{, \the\firstname\the\lastname}
\gdef\formotherauthor{ and \the\firstname\the\lastname}
\gdef\formlastauthor{,\formotherauthor}

\ifechobib{\asis{\font\cmsc=cmcsc10}%%%% small caps font
\gdef\echostart{\ww{}%
\ifx\rl\undefined\def\rll{\the\refnum}\else\def\rll{\rl}\fi%
\ww{\noexpand\ref \rll \the\bsbs}}}\fi

\gdef\formB{\nextref{\cmsc\the\au}, {\it\the\ti\/}, \the\pb, \the\pl, \yr.
\ifechobib{\echostart%
\ww{{\noexpand\cmsc\the\au},}%
\ww{{\noexpand\it\the\ti\noexpand\/},}%
\ww{\the\pb,}%
\ww{\the\pl,}%
\ww{\yr.}%
\ww{\noexpand\endref}}\fi}

\gdef\formD{\nextref{\cmsc\the\au}, ``\the\ti'', dissertation, \the\pl, \yr.
\ifechobib{\echostart%
\ww{{\noexpand\cmsc\the\au},}%
\ww{``\the\ti'',}%
\ww{Ph.D. thesis,}%
\ww{\the\pl,}%
\ww{\yr.}
\ww{\noexpand\endref}}\fi}

\gdef\formJ{\nextref{\cmsc\the\au}, {\it\the\ti\/}, \the\jr, \vl\ (\yr), pp.~\pp.
\ifechobib{\echostart%
\ww{{\noexpand\cmsc\the\au},}%
\ww{{\noexpand\it\the\ti\noexpand\/},}%
\ww{\the\jr, \vl\ (\yr), pp.\noexpand~\pp.}
\ww{\noexpand\endref}}\fi}

\gdef\formP{\nextref{\cmsc\the\au}, {\it\the\ti\/}, \the\tit\ 
(\the\aut, ed\edsop.), \the\pub, \the\pl, \yr, pp.~\pp. 
\ifechobib{\echostart%
\ww{{\noexpand\cmsc\the\au},}%
\ww{{\noexpand\it\the\ti\noexpand\/},}%
\ww{in \the\tit,}%
\ww{\the\aut, ed\edsop,}%
\ww{\the\pub,}%
\ww{\the\pl,}%
\ww{\yr, pp.\noexpand~\pp.}
\ww{\noexpand\endref}}\fi}

\gdef\formR{\nextref{\cmsc\the\au}, {\it\the\ti\/}, \ifx\is\empty\else\is, \fi \yr.
\ifechobib{\echostart%
\ww{{\noexpand\cmsc\the\au},}%
\ww{{\noexpand\it\the\ti\noexpand\/},}%
\ifx\is\empty\ww{\yr.}\else\ww{\is, \yr.}\fi%
\ww{\noexpand\endref}}\fi}
}\fi %%%%%%%%%%%%%%%%%%%%%%%%%%%%%%%%%%% end of definition of the SIAM style

\ifx\CAstyle\undefined\else{ %%%%%%%%%%%%% the Constructive Approximation style
% 10 nov 95 Shayne Waldron; 07 dec 95 sw
%% if the article has not appeared, then the year should be given as 'to appear'

\ifechobib{\asis{\font\cmsc=cmcsc10}}\fi %%%% small caps font

\gdef\formfirstauthor{\the\firstname\the\lastname}
\gdef\formnextauthor{, \the\firstname\the\lastname}
\gdef\formotherauthor{, \the\firstname\the\lastname}
\gdef\formlastauthor{\formotherauthor}

\gdef\formB{\nextref{\cmsc\the\au} (\yr): \the\ti. \the\pl: \the\pb.
\ifechobib{\echostart
\ww{{\noexpand\cmsc\the\au} }
\ww{(\yr):}
\ww{\the\ti.}
\ww{\the\pl:}
\ww{\the\pb.}
}\fi}
\gdef\formD{\nextref{\cmsc\the\au} (\yr): \the\ti. dissertation, \the\pl.
\ifechobib{\echostart
\ww{{\noexpand\cmsc\the\au} }
\ww{(\yr):}
\ww{\the\ti.}
\ww{dissertation,}
\ww{\the\pl.}
}\fi}
\gdef\formJ{\nextref{\cmsc\the\au} (\yr): {\it \the\ti\/}. \the\jr, {\bf\vl}:\pp.
\ifechobib{\echostart
\ww{{\noexpand\cmsc\the\au} }
\ww{(\yr):}
\ww{{\noexpand\it \the\ti\noexpand\/}.}
\ww{\the\jr, {\noexpand\bf\vl}:\pp.}
}\fi}
\gdef\formP{\nextref{\cmsc\the\au} (\yr): {\it \the\ti\/}. In: \the\tit\ ({\cmsc\the\aut}, ed\edsop). \the\pl: \the\pub, pp.~\pp.
\ifechobib{\echostart
\ww{{\noexpand\cmsc\the\au} }
\ww{(\yr):}
\ww{{\noexpand\it \the\ti\noexpand\/}.}
\ww{In: \the\tit\noexpand\ }
\ww{({\noexpand\cmsc\the\aut}, ed\edsop).}
\ww{\the\pl: }
\ww{\the\pub, pp.\noexpand~\pp.}
}\fi}
\gdef\formR{\nextref{\cmsc\the\au} (\yr): {\it \the\ti\/}. \is
\ifechobib{\echostart
\ww{{\noexpand\cmsc\the\au} }
\ww{(\yr):}
\ww{{\noexpand\it \the\ti\noexpand\/}.}
\ww{\is}
}\fi}
}\fi %%%%%%%%%%%%%%%%%%%%%%%%%%%%%% end of the Constructive Approximation style
 
\ifx\MMMASstyle\undefined\else{ %%%%the Math. Models Methods Applied Sci. style 
% cb 2mar00
\gdef\formfirstauthor{\the\firstname\the\lastname}
\gdef\formnextauthor{, \the\firstname\the\lastname}
\gdef\formotherauthor{ and \the\firstname\the\lastname}
\gdef\formlastauthor{,\formotherauthor}

\gdef\formB{\nextref\the\au, {\ninebf\the\ti} (\the\pb, \yr).
\ifechobib{\echostart
\ww{\the\au,}
\ww{{\noexpand\ninebf\the\ti}}
\ww{(\the\pb, \yr).}
}\fi}
\gdef\formD{\nextref\the\au, {\nineit\the\ti} (dissertation, \the\pl, \yr).
\ifechobib{\echostart
\ww{\the\au,}
\ww{{\noexpand\nineit\the\ti}}
\ww{ (dissertation, \the\pl, \yr).}
}\fi}
\gdef\formJ{\nextref\the\au, {\nineit\the\ti}, {\nineit\the\jr}
 {\ninebf\vl} (\yr) \pp.
\ifechobib{\echostart
\ww{\the\au,}
\ww{{\noexpand\nineit\the\ti},}
\ww{{\noexpand\nineit\the\jr}\noexpand\ {\noexpand\ninebf\vl} (\yr) \pp.}
}\fi}
\gdef\formP{\nextref\the\au, {\nineit\the\ti}, in {\ninebf\the\tit}, ed\edsop. \the\aut\ (\the\pub, \yr), pp. \pp. 
\ifechobib{\echostart
\ww{\the\au}
\ww{{\noexpand\nineit\the\ti},}
\ww{in\noexpand\ }
\ww{{\noexpand\ninebf\the\tit},}
\ww{ed\edsop. \the\aut\ (\the\pub, \yr),}
\ww{pp. \pp.}
}\fi}
\gdef\formR{\nextref\the\au, {\nineit\the\ti},  \is\ (\yr).
\ifechobib{\echostart
\ww{\the\au}
\ww{{\noexpand\nineit\the\ti},}
\ww{\is (\yr).}
}\fi}
}\fi %%%%%%%%%%%%%%%% end of the Math. Models Methods in Applied Science style

\ifx\NMstyle\undefined\else{ %%%%%%%%%%%%%%% the Numerische Mathematik style
%%    jun 95/adapted from shayne waldron's missive/cb

\gdef\formfirstauthor{\the\lastname, \the\firstname}
\gdef\formnextauthor{, \the\firstname\the\lastname}
\gdef\formotherauthor{ and \the\firstname\the\lastname}
\gdef\formlastauthor{,\formotherauthor}

\gdef\formB{\nextref\the\au (\yr): \the\ti. \the\pb, \the\pl
\ifechobib{\echostart
\ww{\the\au}
\ww{(\yr):}
\ww{\the\ti.}
\ww{\the\pb,}
\ww{\the\pl}
}\fi}
\gdef\formD{\nextref\the\au (\yr): \the\ti. dissertation, \the\pl
\ifechobib{\echostart
\ww{\the\au}
\ww{(\yr):}
\ww{\the\ti.}
\ww{dissertation,}
\ww{\the\pl}
}\fi}
\gdef\formJ{\nextref\the\au (\yr): \the\ti. \the\jr\ {\bf\vl}, \pp
\ifechobib{\echostart
\ww{\the\au}
\ww{(\yr):}
\ww{\the\ti.}
\ww{\the\jr\noexpand\ {\noexpand\bf\vl}, \pp}
}\fi}
\gdef\formP{\nextref\the\au (\yr): \the\ti. In: \the\aut, ed\edsop, \the\tit,
           \pp. \the\pub, \the\pl
\ifechobib{\echostart
\ww{\the\au}
\ww{(\yr):}
\ww{\the\ti.}
\ww{In:}
\ww{\the\aut, ed\edsop,}
\ww{\the\tit,}
\ww{\pp.}
\ww{\the\pub,}
\ww{\the\pl}
}\fi}
\gdef\formR{\nextref\the\au (\yr): \the\ti. \is
\ifechobib{\echostart
\ww{\the\au}
\ww{(\yr):}
\ww{\the\ti.}
\ww{\is}
}\fi}
}\fi %%%%%%%%%%%%%%%%%%%%%%%%%%%%%%%% end of the Numerische Mathematik style

%%%%%%%%%%%%%%%%%%%%%%%%%%%%%%%%%%%%%%  Eastern Journal of Approximation style
\ifx\EJAstyle\undefined\else{
%%    jan 96/adapted from shayne waldron's missive/cb
\ifechobib{
\asis{\font\cmsc=cmcsc10} %%%% small caps font
}\fi
\gdef\formfirstauthor{\the\firstname\the\lastname}
\gdef\formnextauthor{, \the\firstname\the\lastname}
\gdef\formotherauthor{ and \the\firstname\the\lastname}
\gdef\formlastauthor{\formotherauthor}
\gdef\formB{\nextref{\cmsc\the\au}, \the\ti, \the\pb, \the\pl, \yr.
\ifechobib{\echostart
\ww{{\noexpand\cmsc\the\au},}
\ww{\the\ti,}
\ww{\the\pb,}
\ww{\the\pl,}
\ww{\yr.}
}\fi}
\gdef\formD{\nextref{\cmsc\the\au}, \the\ti, dissertation, \the\pl, \yr.
\ifechobib{\echostart
\ww{{\noexpand\cmsc\the\au},}
\ww{\the\ti,}
\ww{dissertation,}
\ww{\the\pl,}
\ww{\yr.}
}\fi}
\gdef\formJ{\nextref{\cmsc\the\au}, {\the\ti}, {\it\the\jr\/} {\bf\vl}(\yr),
\pp.
\ifechobib{\echostart
\ww{{\noexpand\cmsc\the\au},}
\ww{\the\ti,}
\ww{{\noexpand\bf\vl}(\yr),}
\ww{\pp.}
}\fi}
\gdef\formP{\nextref{\cmsc\the\au}, \the\ti, In: \the\tit\
(\the\aut, Ed\edsop), pp.~\pp, \the\pub, \the\pl, \yr.
\ifechobib{\echostart
\ww{{\noexpand\cmsc\the\au},}
\ww{\the\ti,}
\ww{In: \the\tit\noexpand\ }
\ww{({\the\aut}, Ed\edsop),}
\ww{pp.\noexpand~\pp,}
\ww{\the\pub,} 
\ww{\the\pl,}
\ww{\yr.}
}\fi}
\gdef\formR{\nextref{\cmsc\the\au}, \the\ti, \ifx\is\empty\else \is, \fi \yr.
\ifechobib{\echostart
\ww{{\noexpand\cmsc\the\au},}
\ww{\the\ti,}
\ifx\is\empty\ww{\yr.}\else\ww{\is, \yr.}\fi
}\fi}
}\fi
%%% It seems that they would not include the name of the editor
%%% of a proceedings that an article appears in (unlike Shayne did above!)
%%%%%%%%%%%%%%%%%%%%%%%%%%%%%%%  end of Eastern Journal of Approximation style

\ifx\JATstyle\undefined\else{ %%%%%%%%%%%%%%% J. Approx. Theory style
% cb 20mar97
\gdef\formfirstauthor{\the\firstname\the\lastname}
\gdef\formnextauthor{, \the\firstname\the\lastname}
\gdef\formotherauthor{ and \the\firstname\the\lastname}
\gdef\formlastauthor{,\formotherauthor}
%
\gdef\formB{\nextref\the\au, ``\the\ti'', \the\pb, \the\pl, \yr.
\ifechobib{\echostart
\ww{\the\au,}
\ww{``\the\ti'',}
\ww{\the\pb,}
\ww{\the\pl,}
\ww{\yr.}
}\fi}
\gdef\formD{\nextref\the\au, ``\the\ti'', dissertation, \the\pl, \yr.
\ifechobib{\echostart
\ww{\the\au,}
\ww{``\the\ti'',}
\ww{dissertation,}
\ww{\the\pl,}
\ww{\yr.}
}\fi}
\gdef\formJ{\nextref\the\au, \the\ti, {\it\the\jr\/} {\bf\vl} (\yr), \pp.
\ifechobib{\echostart
\ww{\the\au,}
\ww{\the\ti,}
\ww{{\noexpand\it\the\jr\noexpand\/}}
\ww{{\noexpand\bf\vl}}
\ww{(\yr),}
\ww{\pp.}
}\fi}
\gdef\formP{\nextref\the\au, \the\ti, {\it in\/} ``\the\tit'' 
(\the\aut{} Ed\edsop.), pp.~\pp, \the\pub, \the\pl, \yr.
\ifechobib{\echostart
\ww{\the\au,}
\ww{\the\ti,}
\ww{{\noexpand\it in\noexpand\/}}
\ww{``\the\tit''}
\ww{(\the\aut{} Ed\edsop.),}
\ww{pp.\noexpand~\pp,}
\ww{\the\pub,}
\ww{\the\pl,}
\ww{\yr.}
}\fi}
\gdef\formR{\nextref\the\au, \the\ti, \ifx\is\empty\else\is, \fi \yr.
\ifechobib{\echostart
\ww{\the\au,}
\ww{\the\ti,}
\ifx\is\empty\else\ww{\is,}\fi
\ww{\yr.}
}\fi}
}\fi %%%%%%%%%%%%%%%%%%%%%%%%%%%%%%%%%%%%%%%%%   end of J. Approx. Theory style

\newcount\refitem\refitem0
\ifx\Baltzerstyle\undefined\else{ %%%%%%%%%%%%%%%%%%%%%%  Baltzer style
%
%  references more or less in the style used in Baltzer journals.
%
% 19 jan 96/cb
%
\gdef\nextref{\advance\refitem1\item{[\the\refitem]}}
\ifechobib{\gdef\startecho{\ww{}\ww{\noexpand\item{[\the\refitem]}}}}\fi

% Specify here the particular form(s) for the author field:

\gdef\formfirstauthor{\the\firstname\the\lastname}
\gdef\formnextauthor{, \the\firstname\the\lastname}
\gdef\formotherauthor{ and \the\firstname\the\lastname}
\gdef\formlastauthor{,\formotherauthor}

% Specify here the five basic reference types:

\gdef\formB{\nextref\the\au, {\it \the\ti}, \the\pb, \the\pl, \yr.
\ifechobib{\startecho
\ww{\the\au,}%
\ww{{\noexpand\it \the\ti},}%
\ww{\the\pb,}
\ww{\the\pl,}
\ww{\yr.}
}\fi}

\gdef\formD{\nextref\the\au, {\it \the\ti}, Ph.D.\ thesis, \the\pl, \yr.
\ifechobib{\startecho
\ww{\the\au,}%
\ww{{\noexpand\it \the\ti},}%
\ww{Ph.D.\ thesis,}%
\ww{\the\pl,}%
ww{\yr.}
}\fi}

\gdef\formJ{\nextref\the\au, \the\ti, \the\jr, \vl\ (\yr) \pp.
\ifechobib{\startecho
\ww{\the\au,}%
\ww{\the\ti,}%
\ww{\the\jr, \vl\ (\yr) \pp.}}
\fi}

\gdef\formP{\nextref\the\au, \the\ti, in {\it \the\tit}, \the\aut, ed\edsop.,
\the\pub, \the\pl, \yr, pp.\pp.
\ifechobib{\startecho
\ww{\the\au,}%
\ww{\the\ti,}%
\ww{in {\noexpand\it \the\tit},}%
\ww{\the\aut, ed\edsop.,}%
\ww{\the\pub,}%
\ww{\the\pl,}%
\ww{\yr, pp.\pp.}}\fi}

\gdef\formR{\nextref\the\au, \the\ti, \ifx\is\empty\else\is,\fi \yr.
\ww{\the\au,}%
\ww{\the\ti,}%
\ifx\is\empty\else\ww{\is,}\fi%
\ww{\yr.}}
}\fi %%%%%%%%%%%%%%%%%%%%%%  end of Baltzerstyle

\ifx\AMSstyle\undefined\else{ %%%%%%%%%%%%%%%%%%%%%%       AMSstyle
% 20 june 95/cb
%
\gdef\nextref{\advance\refitem1\item{[\the\refitem]}}
\ifechobib{\gdef\echostart{\ww{}\ww{\noexpand\ref\the\refitem}}}\fi

% Specify here the particular form(s) for the author field:

\gdef\formfirstauthor{\the\firstname\the\lastname}
\gdef\formnextauthor{, \the\firstname\the\lastname}
\gdef\formotherauthor{ and \the\firstname\the\lastname}
\gdef\formlastauthor{,\formotherauthor}

% Specify here the five basic reference types:

\gdef\formB{\nextref \the\au, {\it\the\ti\/}, \the\pb, \the\pl, \yr.%
\ifechobib{\echostart
\ww{\noexpand\by \the\au}%
\ww{\noexpand\book \the\ti}%
\ww{\noexpand\publ \the\pb}
\ww{\noexpand\publaddr \the\pl}
\ww{\noexpand\yr \yr}
\ww{\noexpand\endref}}\fi}

\gdef\formD{\nextref \the\au, {\it\the\ti\/}, Dissertation, \the\pl, \yr.%
\ifechobib{\echostart
\ww{\noexpand\by \the\au}%
\ww{\noexpand\book \the\ti}%
\ww{\noexpand\publ Dissertation}%
\ww{\noexpand\publaddr \the\pl}%
\ww{\noexpand\yr \yr}%
\ww{\noexpand\endref}}\fi}

\gdef\formJ{\nextref \the\au, {\it\the\ti\/}, \the\jr\ (\yr), \pp.%
\ifechobib{\echostart
\ww{\noexpand\by \the\au}%
\ww{\noexpand\paper \the\ti}%
\ww{\noexpand\jour \the\jr}%
\ww{\noexpand\vol \vl}%
\ww{\noexpand\yr \yr}%
\ww{\noexpand\pages \pp}%
\ww{\noexpand\endref}}\fi}

\gdef\formP{\nextref \the\au, {\it\the\ti\/}, \the\tit (\the\aut, ed\edsop.),
\ifechobib{\echostart
\yr, \the\pub, \the\pl, pp.~\pp.,%
\ww{\noexpand\by \the\au}%
\ww{\noexpand\paper \the\ti}%
\ww{\noexpand\inbook \the\tit}%
%\ww{\noexpand\editor \the\aut}%it seems that this is NOT part of AMS style!
\ww{\noexpand\publ\noexpand\ \the\pub}%
\ww{\noexpand\publaddr \the\pl}%
\ww{\noexpand\yr \yr}%
\ww{\noexpand\pages \pp}%
\ww{\noexpand\endref}}\fi}

\gdef\formR{\nextref \the\au, {\it\the\ti\/}, \ifx\is\empty\else{\it\is\/}, \fi
\yr.%
\ifechobib{\echostart
\ww{\noexpand\by \the\au}%
\ww{\noexpand\paper \the\ti}%
\ifx\is\empty\else\ww{\noexpand\publaddr \is}\fi%
\ww{\noexpand\yr \yr}%
\ww{\noexpand\endref}}\fi}
}\fi %%%%%%%%%%%%%%%%%%%%%%  end of AMSstyle

\ifx\Chamonixstyle\undefined\else{ %%%%%%%%%%%%%%%%%%%%  Chamonixstyle
%
%  references in the style of the Chamonix Conference proceedings
%     7 jun 96 Larry Schumaker
%
\gdef\nextref{\ifx\rl\undefined%
 \global\advance\refnum by 1\relax\def\rll{\the\refnum}\else\def\rll{\rl}\fi%
 \smallskip\item{\rll .}}
\ifechobib{\gdef\echostart{\ww{}\ww{\noexpand\ref \rll\the\bsbs}}%
\asis{\font\cmsc=cmcsc10}%%%% small caps font
}\fi


\gdef\formfirstauthor{\the\firstname\the\lastname}
\gdef\formnextauthor{, \the\firstname\the\lastname}
\gdef\formotherauthor{ and \the\firstname\the\lastname}
\gdef\formlastauthor{,\formotherauthor}

\gdef\formB{\nextref{\cmsc\the\au}, {\it\the\ti\/}, \the\pb, \the\pl, \yr.
\ifechobib{\echostart
\ww{{\noexpand\cmsc\the\au},}%
\ww{{\noexpand\it\the\ti\noexpand\/},}%
\ww{\the\pb,}%
\ww{\the\pl,}%
\ww{\yr.}}\fi}

\gdef\formD{\nextref{\cmsc\the\au}, ``\the\ti'', Ph.D. thesis, \the\pl, \yr.
\ifechobib{\echostart
\ww{{\noexpand\cmsc\the\au},}%
\ww{``\the\ti'',}%
\ww{Ph.D. thesis,}%
\ww{\the\pl,}%
\ww{\yr.}}\fi}

\gdef\formJ{\nextref{\cmsc\the\au}, {\it\the\ti\/}, \the\jr, \vl\ (\yr), pp.~\pp.
\ifechobib{\echostart
\ww{{\noexpand\cmsc\the\au},}%
\ww{{\noexpand\it\the\ti\noexpand\/},}%
\ww{\the\jr, \vl\noexpand\ (\yr), pp.\noexpand~\pp.}}\fi}

\gdef\formP{\nextref{\cmsc\the\au}, {\it\the\ti\/}, in \the\tit, \the\aut,
ed\edsop, \the\pub, \the\pl, \yr, pp.~\pp. 
\ifechobib{\echostart
\ww{{\noexpand\cmsc\the\au},}%
\ww{{\noexpand\it\the\ti\noexpand\/},}%
\ww{in \the\tit,}%
\ww{\the\aut, ed\edsop,}%
\ww{\the\pub,}%
\ww{\the\pl,}%
\ww{\yr, pp.\noexpand~\pp.}}\fi}

\gdef\formR{\nextref{\cmsc\the\au}, {\it\the\ti\/}, \ifx\is\empty\else\is, \fi\yr.
\ifechobib{\echostart
\ww{{\noexpand\cmsc\the\au},}%
\ww{{\noexpand\it\the\ti\noexpand\/},}%
\ifx\is\empty\ww{\yr.}\else\ww{\is, \yr.}\fi}\fi}
}\fi %%%%%%%%%%%%%%%%%%%%%%  end of Chamonixstyle

\ifx\BibTeX\undefined\else{ %%%%%%%%%%%%%%%%%%%%%  BibTeX
%
%  I followed the BibTeX style as outlined in Lamport's LaTeX, 1986
%     30nov96 cb
%%%%%%%%%%%%   \startbibout %%%%%%%%%%%%%  
\message{no references will be TeXed; all references will be put}
\message{ in BibTeX format into the file \jobname.bib .}
\gdef\%#1{\oneline={#1}\ww{\the\percent\the\oneline}}

%% the next definitions try to make use of the macro \rhl if it is there
% to fill in the key field, and provide a default fill-in if \rhl is not around.
\gdef\rhl#1{\rahel#1::/\oneline=\rh} % def of \rh in \rahel is subtly different
\gdef\rahel#1:#2:#3/{\edef\rh{{#1}}\def\next{#2}\ifx\next\empty\edef\rl{#1}\else\edef\rl{#2}\fi}
\edef\rh\global\oneline=\rh

% Specify here the particular form(s) for the author field:

\gdef\formfirstauthor{\the\lastname, \the\firstname}
\gdef\formnextauthor{ and\formfirstauthor}
\gdef\formotherauthor{\formnextauthor}
\gdef\formlastauthor{\formnextauthor}

% Specify here the five basic reference types:
%  (note the use of parentheses to delimit the argument for @<type> )

\gdef\formB{\nextref%
\ww{@BOOK(\the\oneline,}
\ww{     AUTHOR = "\the\au",}
\ww{     TITLE  = "\the\ti",}
\ww{     PUBLISHER = "\the\pb",}
\ww{     ADDRESS = "\the\pl")}}

\gdef\formD{\nextref%
\ww{@PHDTHESIS(\the\oneline,}
\ww{   AUTHOR="\the\au",}
\ww{   YEAR="\yr",}
\ww{   TITLE="\the\ti",}
\ww{   SCHOOL="\the\pl")}}

\gdef\formJ{\nextref%
\ww{@ARTICLE(\the\oneline,}
\ww{   AUTHOR="\the\au",}
\ww{   YEAR="\yr",}
\ww{   TITLE="\the\ti",}
\ww{   JOURNAL="\the\jr",}
\ww{   VOLUME="\vl",}
\ww{   PAGES="\pp")}}
 
\gdef\formP{\nextref%
\ww{@INPROCEEDINGS(\the\oneline,}
\ww{   AUTHOR="\the\au",}
\ww{   YEAR="\yr",}
\ww{   TITLE="\the\ti",}
\ww{   PAGES="\pp",}
\ww{   BOOKTITLE="\the\tit",}
\ww{   EDITOR="\the\aut",}
\ww{   PUBLISHER="\the\pub"}
\ww{   ADDRESS="\the\pl")}}

\gdef\formR{\nextref%
\ww{@TECHREPORT(\the\oneline,}
\ww{   AUTHOR="\the\au",}
\ww{   YEAR="\yr",}
\ww{   TITLE="\the\ti"}
\ifx\is\empty\else\ww{   INSTITUTION="\is"\fi)}}
}\fi %%%%%%%%%%%%%%%%%%%%%%%%%%%%%%%%%%%%%  end of BibTeX

\ifx\plainstyle\undefined\else{%%%%%%%%%%%%%%% This defines a plain text style
%%    jun 97

\gdef\formfirstauthor{\the\firstname\the\lastname}
\gdef\formnextauthor{, \the\firstname\the\lastname}
\gdef\formotherauthor{ and \the\firstname\the\lastname}
\gdef\formlastauthor{,\formotherauthor}

\ifechobib{
\gdef\echostart{\ww{}\global\advance\refnum by 1\ww{\the\refnum}}}\fi

\gdef\formB{\nextref\the\au, \the\ti, \the\pb, \the\pl, \yr.
\ifechobib{\echostart%
\ww{\the\au,}%
\ww{\the\ti,}%
\ww{\the\pb,}%
\ww{\the\pl,}%
\ww{\yr.}}\fi}

\gdef\formD{\nextref\the\au, \the\ti, dissertation, \the\pl, \yr.
\ifechobib{\echostart%
\ww{\the\au,}%
\ww{\the\ti,}%
\ww{Ph.D. thesis,}%
\ww{\the\pl,}%
\ww{\yr.}}\fi}

\gdef\formJ{\nextref\the\au, \the\ti, \the\jr\ \vl, \pp\ (\yr).
\ifechobib{\echostart%
\ww{\the\au,}%
\ww{\the\ti,}%
\ww{\the\jr\ \vl, \pp\ (\yr).}}\fi}

\gdef\formP{\nextref\the\au, \the\ti, in \the\tit\ 
(\the\aut, editor\edsop), \the\pub (\the\pl), \pp, (\yr). 
\ifechobib{\echostart%
\ww{\the\au,}%
\ww{\the\ti,}%
\ww{in \the\tit}%
\ww{(\the\aut, editor\edsop),}%
\ww{\the\pub (\the\pl),}%
\ww{\pp, (\yr).}}\fi}

\gdef\formR{\nextref\the\au, \the\ti, \is, \yr.
\ifechobib{\echostart%
\ww{\the\au,}%
\ww{\the\ti,}%
\ifx\is\empty\ww{\yr.}\else\ww{\is, \yr.}\fi%
\ww{\noexpand\endref}}\fi}
}\fi %%%%%%%%%%%%%%%%%%%%%%%%%%%%%%%%%%%%%  end of plain text style
\ifx\Kluwerstyle\undefined\else{%%%%%%%%%%%%%%% This defines Kluwer style
%%    feb 04

\gdef\formfirstauthor{\the\lastname, \the\firstname}
\gdef\formnextauthor{, \the\firstname\the\lastname}
\gdef\formotherauthor{ and \the\firstname\the\lastname}
\gdef\formlastauthor{,\formotherauthor}

\ifechobib{
\gdef\echostart{\ww{}
\ww{\noexpand\bibitem[\noexpand\protect\noexpand\citeauthoryear{\the\au}{\yr}]{???}}}}\fi

\gdef\formB{\nextref\the\au, \the\ti, \the\pb, \the\pl, \yr.
\ifechobib{\echostart%
\ww{\the\au.}%
\ww{\noexpand\newblock{\noexpand\em \the\ti}.}%
\ww{\noexpand\newblock \the\pb, \the\pl, \yr.}}\fi}

\gdef\formD{\nextref\the\au, \the\ti, dissertation, \the\pl, \yr.
\ifechobib{\echostart%
\ww{\the\au.}%
\ww{\noexpand\newblock {\noexpand\em \the\ti}.}%
\ww{PhD thesis, \the\pl, \yr.}}\fi}%

\gdef\formJ{\nextref\the\au, \the\ti, \the\jr\ \vl, \pp\ (\yr).
\ifechobib{\echostart%
\ww{\the\au.}%
\ww{\noexpand\newblock \the\ti.}%
\ww{\noexpand\newblock {\noexpand\em \the\jr}, \vl:\pp, \yr.}}\fi}

\gdef\formP{\nextref\the\au, \the\ti, in \the\tit\ 
(\the\aut, editor\edsop), \the\pub (\the\pl), \pp, (\yr). 
\ifechobib{\echostart%
\ww{\the\au.}%
\ww{\noexpand\newblock \the\ti.}%
\ww{\noexpand\newblock In \the\aut, editor\edsop, {\noexpand\em \the\tit}, pages \pp.}%
\ww{\the\pub, \the\pl, \yr.}}\fi}

\gdef\formR{\nextref\the\au, \the\ti, \is, \yr.
\ifechobib{\echostart%
\ww{\the\au.}%
\ww{\noexpand\newblock {\the\ti}.}%
\ifx\is\empty\ww{\yr.}\else\ww{\is, \yr.}\fi%
}\fi}
}\fi %%%%%%%%%%%%%%%%%%%%%%%%%%%%%%%%%%%%%  end of Kluwer style
